\section{Related Work}

DBL draws from and extends work in AI safety, formal methods, information flow control, and audit systems. DBL is positioned relative to five categories of prior work.

\subsection{AI Safety and Governance}

\paragraph{Constitutional AI and RLHF.}
Constitutional AI~\cite{bai2022constitutional} and RLHF~\cite{ouyang2022training} align model behavior via iterative optimization over outputs and preference signals. This couples governance with execution outcomes and yields policies that are not replay-deterministic across training runs. DBL instead externalizes normativity as explicit, pre-execution DECISION events and treats outputs as observational.

\paragraph{Guardrails and Prompt Filtering.}
Guardrail systems~\cite{rebedea2023nemo,inan2023llama} filter prompts and responses against safety policies, typically as pre-processing or post-processing layers. Some guardrails may adapt using observed violations, and logging is frequently incomplete. DBL requires explicit DECISION events before execution and enforces observational non-interference.

\paragraph{Red Teaming and Adversarial Testing.}
Red teaming~\cite{perez2022red,ganguli2022red} probes model behavior with adversarial prompts to discover failure modes. Used for safety evaluation, not runtime governance.

DBL distinction: Red teaming is evaluation; DBL is enforcement. Red teaming identifies weaknesses; DBL enforces decisions before execution.

\subsection{Formal Methods and Verification}

\paragraph{TLA+ and State Machines.}
TLA+~\cite{lamport2002specifying} specifies distributed systems as state machines with temporal properties. Enables verification of safety and liveness properties.

TLA+ focuses on consensus and distributed correctness; DBL focuses on normative decisions under non-deterministic execution. TLA+ verifies that all executions satisfy invariants; DBL ensures decisions are independent of execution outcomes. DBL uses an explicit normative vs observational separation that is not required by TLA+.

\paragraph{Event Sourcing and CQRS.}
Event sourcing~\cite{fowler2005event} treats events as the source of truth. Commands produce events; queries read projections. CQRS (Command Query Responsibility Segregation) separates write and read models.

Event sourcing reconstructs state by replaying events and treating all events as state-changing. DBL does not require event sourcing but separates normative replay from observational recording and uses that separation as the determinism and replay boundary. DBL also enforces pre-execution decisions and G/L separation.

\subsection{Information Flow Control}

\paragraph{Decentralized Information Flow Control (DIFC).}
DIFC systems~\cite{myers1999jflow,zeldovich2006making} enforce confidentiality and integrity through labeled data and security policies. Information flow is tracked statically or dynamically.

DIFC controls data flows for confidentiality and integrity; DBL controls normative influence for auditability. Observational non-interference aligns with classical noninterference formulations in secure systems~\cite{goguen1982security,rushby1992noninterference}, where high-integrity decision logic must be insulated from low-integrity observations. DBL adds normative primacy (DECISION events) and replay equivalence.

\paragraph{Reference Monitors.}
Reference monitors~\cite{anderson1972reference} mediate access to resources. Enforce security policies at runtime.

Reference monitors enforce access control during execution. DBL decides before execution and excludes observational context from decisions, which yields deterministic governance and replayable normativity.

\subsection{Policy Languages and Access Control}

\paragraph{XACML and ABAC.}
XACML~\cite{oasis2013xacml} is an XML-based policy language for attribute-based access control (ABAC). Policies evaluate attributes of subject, resource, action, and environment.

XACML allows environment attributes (time, location) as decision inputs and does not separate observational data by default. DBL relocates all decision-relevant context into a versioned admission boundary, making the input set itself auditable and replay-stable.

\subsection{Audit and Compliance Systems}

\paragraph{Blockchain and Immutable Logs.}
Blockchain systems~\cite{nakamoto2008bitcoin} provide append-only, tamper-evident logs. Used for audit trails and provenance.

Blockchain focuses on consensus and immutability. DBL focuses on normative decisions under non-deterministic execution and separates normative from observational events while remaining replayable.

\paragraph{Runtime Verification.}
Runtime verification~\cite{leucker2009brief} monitors executions against formal specifications. Detects violations at runtime.

Runtime verification is reactive and observes execution traces; DBL is proactive and decides before execution. Observability and monitoring frameworks record traces and metrics but do not define normative decision boundaries. DBL separates normative decisions from observations and enforces deterministic governance.

\subsection{Summary Table}

\begin{table}[h]
\centering
\small
\begin{tabular}{|l|l|l|l|l|}
\hline
\textbf{Category} & \textbf{Normative} & \textbf{Observational} & \textbf{Replay /} & \textbf{DBL Delta} \\
\textbf{} & \textbf{Primitive} & \textbf{Handling} & \textbf{Reconstruction} & \textbf{} \\
\hline
Constitutional AI & Principles/ & Outputs drive & Non-det & Pre-exec DECISION \\
& Rewards & refinement & & Normative vs obs \\
\hline
Guardrails & Filter rules & May adapt & Limited & Pre-exec DECISION \\
& & from outputs & logging & Non-interference \\
\hline
TLA+ & State & External to & Trace & Normative vs obs \\
& transitions & spec & exploration & Separation boundary \\
\hline
Event Sourcing & Events & Projections & Deterministic & Normative vs obs \\
& & descriptive & replay & Replay boundary \\
\hline
DIFC & Labels, & May carry & Not goal & G/L separation \\
& policies & labels & & Noninterference \\
\hline
Reference & Access & May log & Not & Pre-exec DECISION \\
Monitor & decisions & attempts & guaranteed & Deterministic norm \\
\hline
XACML/ABAC & Policy rules & Time/env & Not & Versioned inputs \\
& & as input & designed & Deterministic G \\
\hline
Blockchain & Transactions & For consensus & Deterministic & Normative vs obs \\
& & not normative & replay & G/L separation \\
\hline
Runtime & Spec & Monitors & Offline & Pre-exec DECISION \\
Verification & violations & traces & verification & Non-interference \\
\hline
\end{tabular}
\caption{Comparison of DBL with related approaches. DBL combines deterministic governance, pre-execution decisions, observational non-interference, and replay equivalence in a single architectural model.}
\label{tab:related-work}
\end{table}

\subsection{Positioning DBL}

DBL is not a replacement for these approaches but an architectural foundation that can compose with them:

\begin{itemize}
  \item \textbf{Constitutional AI / RLHF:} DBL can record policy decisions before invoking constitutionally-aligned models. The model is the effector; DBL governs its invocation.
  
  \item \textbf{Guardrails:} Guardrail rules can be implemented in DBL's boundary layer (L) or governance layer (G). DBL makes their decisions explicit and replayable.
  
  \item \textbf{Event Sourcing:} DBL extends event sourcing by distinguishing normative events (DECISION) from observational events (EXECUTION, PROOF).
  
  \item \textbf{DIFC / Reference Monitors:} DBL's L layer enforces information flow (DIFC-like). DBL's G layer enforces normative decisions (reference monitor-like). DBL adds determinism and replay.
  
  \item \textbf{Audit Systems:} DBL's V stream is an audit log with normative primacy. Blockchain techniques can secure V.
\end{itemize}

The contribution is an architectural separation that makes normativity explicit, deterministic, and replayable even when execution is non-deterministic.

\subsection{Limitations and Future Work}

DBL does not address:
\begin{itemize}
  \item \textbf{Policy quality:} DBL ensures decisions are deterministic and replayable, not correct or optimal.
  \item \textbf{Side channels:} Beyond explicit admission rules, side channels (timing, resource usage) are out of scope.
  \item \textbf{Multi-tenancy:} Federation and cross-tenant governance require extensions to the model.
  \item \textbf{Adaptive policies:} Learning-based or feedback-driven policies violate determinism unless explicitly versioned.
\end{itemize}

Future work includes extending DBL to multi-agent settings, integrating with federated governance, and exploring mechanized verification of DBL properties in proof assistants.
